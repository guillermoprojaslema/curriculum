%% start of file `template.tex'.
%% Copyright 2006-2015 Xavier Danaux (xdanaux@gmail.com).
%
% This work may be distributed and/or modified under the
% conditions of the LaTeX Project Public License version 1.3c,
% available at http://www.latex-project.org/lppl/.



\documentclass[11pt,a4paper,sans]{moderncv}        % possible options include font size ('10pt', '11pt' and '12pt'), paper size ('a4paper', 'letterpaper', 'a5paper', 'legalpaper', 'executivepaper' and 'landscape') and font family ('sans' and 'roman')

\usepackage{tikz}
\usetikzlibrary{shapes.geometric}



\newcommand\score[2]{
\pgfmathsetmacro\pgfxa{#1+1}
\tikzstyle{scorestars}=[star, star points=5, star point ratio=2.25, draw,inner sep=1.3pt,anchor=outer point 3]
  \begin{tikzpicture}[baseline]
    \foreach \i in {1,...,#2} {
    \pgfmathparse{(\i<=#1?"yellow":"gray")}
    \edef\starcolor{\pgfmathresult}
    \draw (\i*1.75ex,0) node[name=star\i,scorestars,fill=\starcolor]  {};
   }
  \end{tikzpicture}
}



% moderncv themes
\moderncvstyle{casual}                             % style options are 'casual' (default), 'classic', 'banking', 'oldstyle' and 'fancy'
\moderncvcolor{blue}                               % color options 'black', 'blue' (default), 'burgundy', 'green', 'grey', 'orange', 'purple' and 'red'
%\renewcommand{\familydefault}{\sfdefault}         % to set the default font; use '\sfdefault' for the default sans serif font, '\rmdefault' for the default roman one, or any tex font name
%\nopagenumbers{}                                  % uncomment to suppress automatic page numbering for CVs longer than one page

% character encoding
\usepackage[utf8]{inputenc}                       % if you are not using xelatex ou lualatex, replace by the encoding you are using
%\usepackage{CJKutf8}                              % if you need to use CJK to typeset your resume in Chinese, Japanese or Korean

% adjust the page margins
\usepackage[scale=0.75]{geometry}
%\setlength{\hintscolumnwidth}{3cm}                % if you want to change the width of the column with the dates
%\setlength{\makecvtitlenamewidth}{10cm}           % for the 'classic' style, if you want to force the width allocated to your name and avoid line breaks. be careful though, the length is normally calculated to avoid any overlap with your personal info; use this at your own typographical risks...

% personal data
\name{Guillermo}{Rojas Lema}
\title{Currículum Vitae}                               % optional, remove / comment the line if not wanted
\address{Pasaje El Partidor 1648}{Puente Alto}{Santiago}% optional, remove / comment the line if not wanted; the "postcode city" and "country" arguments can be omitted or provided empty
\phone[mobile]{+56~(9)~8810~7184}                   % optional, remove / comment the line if not wanted; the optional "type" of the phone can be "mobile" (default), "fixed" or "fax"
\phone[fixed]{+56~(2)~2823~1918}
% \phone[fax]{+3~(456)~789~012}
\email{guillermorojaslema@gmail.com}                               % optional, remove / comment the line if not wanted
%\homepage{www.johndoe.com}                         % optional, remove / comment the line if not wanted
\social[linkedin]{guillermo-rojas-lema-b78580112}                        % optional, remove / comment the line if not wanted
% \social[twitter]{mitorojaslema}                             % optional, remove / comment the line if not wanted
\social[github]{guillermoprojaslema}                              % optional, remove / comment the line if not wanted
%\extrainfo{additional information}                 % optional, remove / comment the line if not wanted
% \photo[64pt][0.4pt]{picture}                       % optional, remove / comment the line if not wanted; '64pt' is the height the picture must be resized to, 0.4pt is the thickness of the frame around it (put it to 0pt for no frame) and 'picture' is the name of the picture file
\quote{Who controls the past controls the future. Who controls the present controls the past}                                 % optional, remove / comment the line if not wanted

% bibliography adjustements (only useful if you make citations in your resume, or print a list of publications using BibTeX)
%   to show numerical labels in the bibliography (default is to show no labels)
\makeatletter\renewcommand*{\bibliographyitemlabel}{\@biblabel{\arabic{enumiv}}}\makeatother
%   to redefine the bibliography heading string ("Publications")
%\renewcommand{\refname}{Articles}

% bibliography with mutiple entries
%\usepackage{multibib}
%\newcites{book,misc}{{Books},{Others}}
%----------------------------------------------------------------------------------
%            content
%----------------------------------------------------------------------------------
\begin{document}
%\begin{CJK*}{UTF8}{gbsn}                          % to typeset your resume in Chinese using CJK
%-----       resume       ---------------------------------------------------------
\makecvtitle

\section{Educación}
\cventry{1994--2002}{Básica}{Escuela Particular Nº3}{Puente Alto, Santiago, Chile}{\textit{Graduado}}{}  % arguments 3 to 6 can be left empty
\cventry{2003--2006}{Media}{Liceo Rojas Tocornal}{Puente Alto, Santiago, Chile}{\textit{Graduado}}{}
\cventry{2003--2004}{Idiomas}{Instituo Chileno-Británico}{La Florida, Santiago, Chile}{\textit{}}{Logré completar 5 cursos de inglés intermedio-avanzando}  % arguments 3 to 6 can be left empty
\cventry{2007--2015}{Superior}{Universidad Tecnológica Metropolitana}{Ñuñoa, Santiago, Chile}{\textit{Título Pregrado}}{Ingeniería en Informática}  % arguments 3 to 6 can be left empty


\section{Experiencia laboral}
\cventry{Abril--Diciembre 2013}{Ayudantías}{UTEM}{Santiago}{Ñuñoa}{Desarrollé ayudantías para la asignatura de ciencia de la computación e Ingeniería de Software, en Universidad Tecnológica Metropolitana}
\cventry{Enero--Marzo 2014}{Programador}{Escaleno}{Santiago}{Providencia}{Desarrollé mi práctica profesional habiendo cumplido 320 horas, en las cuales desarrollé tareas de detección de errores en CMS escritos en PHP. Además de desarrollar técnicas de scraping utilizando Node.js}
\cventry{Mayo--Septiembre 2015}{Programador}{MiningTag}{Santiago}{Providencia}{Trabajé como desarrollador, para minintag.cl, en .NET utilizando C\# bajo WPF, para la creación de un mantenedor que controlara las direcciones IP de máquinas con SO embedidos, en camiones que realizaban trabajos para la minería. Apoyo en UX/UI para la construción de software para la minería de servicio de monitorización y señalética.}
\cventry{Marzo--Junio 2016}{Programador}{IdeaUno}{Santiago}{Providencia}{Trabajé como desarrollador, para IdeaUno, en PHP. Bajo las directrices de maqueteador y mantenimiento front-end. Utilizando herramientas como Bootstrap, Material y SaaS. En algunos proyectos, utilicé CakePHP}
\cventry{Diciembre 2016--Enero 2017}{Programador}{EasyPoint}{Santiago}{Ñuñoa}{Desarrollé ayudantías para la asignatura de ciencia de la computación e Ingeniería de Software, en Universidad Tecnológica Metropolitana}
\cventry{Actualidad}{Programador}{}{}{}{En la actualidad, aprendiendo Ruby on Rails, desarrollando un proyecto de un sitio web, para el rubro del contrstrucción; su administración, sección de noticias, torneos, empleos, etc. Puedes chequearlo \href{https://github.com/guillermoprojaslema/contractor}{\textcolor{blue}{aquí}}} 

\section{Idiomas}
\cvdoubleitem{Castellano}{\score{5}{5}}{Inglés}{\score{4.5}{5}}

\section{Habilidades TI}
\cvdoubleitem{OS}{Linux, Windows}{Administración}{Apache, Puma}
\cvdoubleitem{Programación}{C\#, Processing, Arduino}{Scripting}{PHP, Bash, Ruby}
\cvdoubleitem{Diseño web}{XHTML, Javascript}{Base de datos}{MySQL, Postgres, SQLite}

\section{Intereses}
\cvitem{Ciclismo}{Me gusta andar en bicicleta, salir y sentir el aire que hace contacto con la piel}
\cvitem{Música}{Disfruto de tocar la guitarra, y aprendiendo armónica}
\cvitem{Cine}{De vez en cuando, me siento a ver películas}

\section{Referencias}
\begin{cvcolumns}
  \cvcolumn{}{\begin{itemize}\item Cristian Garrido \item Claudio Acuña \item Sebastián Salazar \item José Valenzuela \item Víctor Ávila \end{itemize}}

%  \cvcolumn{Category 1}{\begin{itemize}\item +56949091795 \item +56990729744 \item +56988911045 \item +56979309032 
%  \item +56951261107 \end{itemize}}

 \cvcolumn{}{\begin{itemize}\item cristian.garrido@ceinf.cl \item claudioacunacuna@gmail.com  \item 
 sebasalazar@gmail.com \item jvalenzuela@miningtag.cl \item victorhugo.avila@easypoint.co \end{itemize}}
  
  
\end{cvcolumns}

\end{document}


%% end of file `template.tex'.
