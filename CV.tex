%% start of file `template.tex'.
%% Copyright 2006-2015 Xavier Danaux (xdanaux@gmail.com).
%
% This work may be distributed and/or modified under the
% conditions of the LaTeX Project Public License version 1.3c,
% available at http://www.latex-project.org/lppl/.



\documentclass[12pt,a4paper,sans]{moderncv}        % possible options include font size ('10pt', '11pt' and '12pt'), paper size ('a4paper', 'letterpaper', 'a5paper', 'legalpaper', 'executivepaper' and 'landscape') and font family ('sans' and 'roman')


% moderncv themes
\moderncvstyle{banking}                             % style options are 'casual' (default), 'classic', 'banking', 'oldstyle' and 'fancy'
\moderncvcolor{blue}                               % color options 'black', 'blue' (default), 'burgundy', 'green', 'grey', 'orange', 'purple' and 'red'
%\renewcommand{\familydefault}{\sfdefault}         % to set the default font; use '\sfdefault' for the default sans serif font, '\rmdefault' for the default roman one, or any tex font name
\nopagenumbers{}                                  % uncomment to suppress automatic page numbering for CVs longer than one page

% character encoding
\usepackage[utf8]{inputenc}                       % if you are not using xelatex ou lualatex, replace by the encoding you are using
%\usepackage{CJKutf8}                              % if you need to use CJK to typeset your resume in Chinese, Japanese or Korean

% adjust the page margins
\usepackage[scale=0.75]{geometry}
%\setlength{\hintscolumnwidth}{3cm}                % if you want to change the width of the column with the dates
%\setlength{\makecvtitlenamewidth}{10cm}           % for the 'classic' style, if you want to force the width allocated to your name and avoid line breaks. be careful though, the length is normally calculated to avoid any overlap with your personal info; use this at your own typographical risks...

% personal data
\name{Guillermo}{Rojas Lema}
%\title{Currículum Vítae}                               % optional, remove / comment the line if not wanted
\address{Pasaje El Partidor 1648}{Puente Alto}{Santiago}% optional, remove / comment the line if not wanted; the "postcode city" and "country" arguments can be omitted or provided empty
\phone[mobile]{+56~(9)~8810~7184}                   % optional, remove / comment the line if not wanted; the optional "type" of the phone can be "mobile" (default), "fixed" or "fax"
\phone[fixed]{+56~(2)~2823~1918}
% \phone[fax]{+3~(456)~789~012}
\email{guillermorojaslema@gmail.com}                               % optional, remove / comment the line if not wanted
%\homepage{www.johndoe.com}                         % optional, remove / comment the line if not wanted
%\social[linkedin]{guillermo-rojas-lema-b78580112}                        % optional, remove / comment the line if not wanted
% \social[twitter]{mitorojaslema}                             % optional, remove / comment the line if not wanted
\social[github]{guillermoprojaslema}                              % optional, remove / comment the line if not wanted
%\extrainfo{additional information}                 % optional, remove / comment the line if not wanted
\photo[64pt][0.4pt]{profile_photo}                       % optional, remove / comment the line if not wanted; '64pt' is the height the picture must be resized to, 0.4pt is the thickness of the frame around it (put it to 0pt for no frame) and 'picture' is the name of the picture file
% \quote{Quien controla el pasado, controla el futuro. Quien controla el presente, controla el pasado (George Orwell)}                                 % optional, remove / comment the line if not wanted

% bibliography adjustements (only useful if you make citations in your resume, or print a list of publications using BibTeX)
%   to show numerical labels in the bibliography (default is to show no labels)
\makeatletter\renewcommand*{\bibliographyitemlabel}{\@biblabel{\arabic{enumiv}}}\makeatother
%   to redefine the bibliography heading string ("Publications")
%\renewcommand{\refname}{Articles}

% bibliography with mutiple entries\addbibresource{CV}

%\usepackage{multibib}
%\newcites{asd,asd}{{Boasdadoks},{Othasdasdaers}}



%----------------------------------------------------------------------------------
%            content
%----------------------------------------------------------------------------------
\begin{document}
%\begin{CJK*}{UTF8}{gbsn}                          % to typeset your resume in Chinese using CJK
%-----       resume       ---------------------------------------------------------
\makecvtitle

\section{Experiencia laboral}

\cventry{Actualidad}{Programador}{Freelance}{Santiago}{Puente Alto}{Desarrollando un sistema de información, enfocado la vitrina de productos del rubro de la gestión inmobiliaria, dentro de una PYME, en PHP bajo el framework de Laravel 5.4. Puedes revisarlo \href{http://makpropiedades.herokuapp.com}{\textcolor{blue}{aquí}}} 



\cventry{Noviembre 2017--Enero 2018}{Programador}{Agéndamed}{Santiago}{Las Condes}{Desarrollo de funcionalidad de sistema de información, enfocado al agendamiento y gestión de horas médicas, enfocado a pacientes, administradores, médicos, de varios centros de salud pública y privada, donde la información es consumida a través de una API, para su posterior uso en una aplicación de Andriod y iOS, en PHP 7.1 bajo el framework de Laravel 5.4.
Programador para Inmotion}

\cventry{Agosto 2017--Octubre 2017}{Programador}{Innovit}{Santiago}{Ñuñoa}{Desarrollo de funcionalidad de sistema de información, enfocado al agendamiento y gestión de horas médicas, de varios centro de salud pública de la comuna de Cabrero, en PHP 7.1 bajo el framework de Laravel 5.2.
Programador para Digital Consulting}


\cventry{Diciembre 2016--Enero 2017}{Programador}{EasyPoint}{Santiago}{Ñuñoa}{Adquirí conocimiento para continuar el desarrollo del Front-End, usando Javascript, bajo el framework de Angular 1.6.X, bajo las directrices de maqueteador y creaciones de nuevos módulos de administración logística y presentación de datos, a base de consultas tipo ajax; Utilizando preprosesadores como Bootstrap, y automatizadores de tareas como Gulp}

\cventry{Marzo--Junio 2016}{Programador}{IdeaUno}{Santiago}{Providencia}{Puse en práctica nuevos flujos de trabajo cono GitFlow. Paralelamente aprendí y apliqué el uso de Sass para el despligue de CSS3. Todo esto bajo el Framework CakePHP, para la creación de software a medida de distintos clientes} %

\cventry{Mayo--Septiembre 2015}{Programador}{MiningTag}{Santiago}{Providencia}{.NET utilizando C\# bajo WPF, para la creación de un mantenedor que controlara las direcciones IP de máquinas con SO embedidos, en camiones que realizaban trabajos para la minería, logrando un desarrollo autónomo de principio a fin}

\cventry{Enero--Marzo 2014}{Práctica Profesional}{Escaleno}{Santiago}{Providencia}{Desarrollé tareas de detección de errores en CMS escritos en PHP. Además de aprender y desarrollar técnicas de scraping utilizando Node.js}

\cventry{Abril--Diciembre 2013}{Ayudantías}{UTEM}{Santiago}{Ñuñoa}{Asignaturas de Ciencias de la Computación e Ingeniería de Software, adquiriendo habilidades de dictar y planificar clases y autoinstruirse en herramientas y software a utilizar con el fin de incrementar el aprendizaje y desarrollo de sistemas de información}


\section{Educación}
\cventry{2007--2015}{Ingeniería en Informática}{Universidad Tecnológica Metropolitana}{Ñuñoa, Santiago, Chile}{\textit{Licenciado en Ciencias de la Ingeniería}}{}  % arguments 3 to 6 can be left empty


\section{Formación extracurricular}
\cventry{2003--2004}{Idiomas}{Instituto Chileno-Británico}{La Florida, Santiago, Chile}{\textit{}}{%Logré completar 5 cursos de inglés intermedio-avanzando}  % arguments 3 to 6 can be left empty
}


\section{Experiencia TI}
\cvdoubleitem{OS}{Linux, Windows}{Administración}{Apache, Puma}
\cvdoubleitem{Diseño web}{XHTML, Javascript}{Database}{MySQL, PostgreSQL, SQLite}
\cvdoubleitem{Programación}{PHP, Ruby, C\#}{Programación}{Bash, Processing, Arduino}
\cvdoubleitem{Frameworks}{Rails, Laravel, CakePHP, Codeigniter}{KDD}{Weka}


%\section{Actualidad}
%\cventry{}{Programador}{}{}{}{Desarrollando un proyecto personal en Laravel. Puedes revisarlo \href{https://github.com/guillermoprojaslema/makpropiedades}{\textcolor{blue}{aquí}}} 


\section{Referencias}
\begin{cvcolumns}
  \cvcolumn{}{\begin{itemize}\item Sebastián Salazar  
  \item Víctor Ávila 
  \item Daniel Quiroz 
  \end{itemize}}

 \cvcolumn{}{\begin{itemize} 
 \item sebasalazar@gmail.com 
 \item victorhugo.avila@easypoint.co
  \item quiroz.daniel@gmail.com
  \end{itemize}}
  %\cvcolumn{}{\begin{itemize}\item +56988911045 \item +56979309032 \item +56951261107 \end{itemize}}
  
\end{cvcolumns}


\end{document}


%% end of file `template.tex'. --force
\grid
